\documentclass[12pt]{article}

\title{MIT 18-S-S996: Category theory for scientists, Spring 2013  \\ Homework and personal notes}
\author{Armand Sumo -- \texttt{armandsumo@gmail.com}}
\usepackage{etex}
\usepackage{savesym}
\usepackage{amssymb, amsmath,amsthm,amscd}
%\savesymbol{lrcorner}
\usepackage{txfonts}
%\usepackage{marvosym}
\usepackage{wasysym}
\savesymbol{Sun}\savesymbol{Mercury}\savesymbol{Venus}\savesymbol{Earth}\savesymbol{Mars}\savesymbol{Jupiter}\savesymbol{Saturn}\savesymbol{Uranus}\savesymbol{Neptune}\savesymbol{Pluto}\savesymbol{leftmoon}\savesymbol{rightmoon}\savesymbol{fullmoon}\savesymbol{newmoon}\savesymbol{Aries}\savesymbol{Taurus}\savesymbol{Gemini}\savesymbol{Leo}\savesymbol{Libra}\savesymbol{Scorpio}\savesymbol{diameter}
\usepackage{mathabx}
%\usepackage{stmaryrd}
\usepackage{setspace}
\usepackage{chngcntr}
\usepackage[tt]{titlepic}
\usepackage{enumerate,makecell}
\usepackage{makeidx,tabularx,dashbox}
\usepackage[usenames,dvipsnames]{xcolor}
\usepackage[bookmarks=true,colorlinks=true, linkcolor=MidnightBlue, citecolor=cyan]{hyperref}
\usepackage{lmodern}
\usepackage{graphicx,float}
\usepackage{multirow}
\usepackage{geometry}
\newgeometry{left=1.6in,right=1.6in,top=1.4in,bottom=1.4in}

\usepackage{color}
\usepackage[all,poly,color,matrix,arrow]{xy}
\makeindex

%\usepackage{showkeys}

\newcommand{\comment}[1]{}

\newcommand{\longnote}[2][4.9in]{\fcolorbox{black}{yellow}{\parbox{#1}{\color{black} #2}}}
\newcommand{\shortnote}[1]{\fcolorbox{black}{yellow}{\color{black} #1}}
\newcommand{\start}[1]{\shortnote{Start here: #1.}}
\newcommand{\q}[1]{\begin{question}#1\end{question}}
\newcommand{\g}[1]{\begin{guess}#1\end{guess}}
\newcommand{\cfbox}[2]{
    \colorlet{currentcolor}{.}
    {\color{#1}
    \fbox{\color{currentcolor}#2}}
}

\def\tn{\textnormal}
\def\mf{\mathfrak}
\def\mc{\mathcal}
\newcommand{\qt}[1]{\tn{``}#1\tn{"}}

\def\ZZ{{\mathbb Z}}
\def\QQ{{\mathbb Q}}
\def\RR{{\mathbb R}}
\def\CC{{\mathbb C}}
\def\AA{{\mathbb A}}
\def\PP{{\mathbb P}}
\def\NN{{\mathbb N}}


\def\Hom{\tn{Hom}}
\def\Path{\tn{Path}}
\def\Paths{\tn{Paths}}
\def\List{\tn{List}}
\def\im{\tn{im}}
\def\Fun{\tn{Fun}}
\def\Ob{\tn{Ob}}
\def\Skel{\tn{Skel}}
\def\Op{\tn{Op}}
\def\PK{\tn{PK}}
\def\FK{\tn{FK}}
\def\SEL*{\tn{SEL*}}
\def\Res{\tn{Res}}
\def\hsp{\hspace{.3in}}
\newcommand{\hsps}[1]{{\hspace{2mm} #1\hspace{2mm}}}
\newcommand{\tin}[1]{\text{\tiny #1}}

\def\singleton{\{\smiley\}}
\newcommand{\boxtitle}[1]{\begin{center}#1\end{center}\vspace{-.1in}}
\newcommand{\singlefun}[1]{\star^{#1}}
\newcommand{\pullb}[1]{\Delta_{#1}}
\newcommand{\lpush}[1]{\Sigma_{#1}}
\newcommand{\rpush}[1]{\Pi_{#1}}
\def\lcone{^\triangleleft}
\def\rcone{^\triangleright}
\def\to{\rightarrow}
\def\from{\leftarrow}
\def\down{\downarrrow}
\def\Down{\Downarrow}
\def\cross{\times}
\def\taking{\colon}
\def\inj{\hookrightarrow}
\def\surj{\twoheadrightarrow}
\def\too{\longrightarrow}
\newcommand{\xyright}[1]{\xymatrix{~\ar[r]#1&}}
\newcommand{\xydown}[1]{\xymatrix{~\ar[d]#1\\~}}
\newcommand{\xydoown}[1]{\xymatrix{~\ar[ddd]#1\\\parbox{0in}{~}\\\parbox{0in}{~}\\~}}
\def\fromm{\longleftarrow}
\def\tooo{\longlongrightarrow}
\def\tto{\rightrightarrows}
\def\ttto{\equiv\!\!>}
\def\ss{\subseteq}
\def\superset{\supseteq}
\def\iso{\cong}
\def\down{\downarrow}
\def\|{{\;|\;}}
\def\m1{{-1}}
\def\op{^\tn{op}}
\def\loc{\tn{loc}}
\def\la{\langle}
\def\ra{\rangle}
\def\wt{\widetilde}
\def\wh{\widehat}
\def\we{\simeq}
\def\ol{\overline}
\def\ul{\underline}
\def\plpl{+\!\!+\hspace{1pt}}
\def\acts{\lefttorightarrow}
\def\vect{\overrightarrow}
\def\qeq{\mathop{=}^?}
\def\del{\partial\,}

\def\rr{\raggedright}

%\newcommand{\LMO}[1]{\bullet^{#1}}
%\newcommand{\LTO}[1]{\bullet^{\tn{#1}}}
\newcommand{\LMO}[1]{\stackrel{#1}{\bullet}}
\newcommand{\LTO}[1]{\stackrel{\tt{#1}}{\bullet}}
\newcommand{\LA}[2]{\ar[#1]^-{\tn {#2}}}
\newcommand{\LAL}[2]{\ar[#1]_-{\tn {#2}}}
\newcommand{\obox}[3]{\stackrel{#1}{\fbox{\parbox{#2}{#3}}}}
\newcommand{\labox}[2]{\obox{#1}{1.6in}{#2}}
\newcommand{\mebox}[2]{\obox{#1}{1in}{#2}}
\newcommand{\smbox}[2]{\stackrel{#1}{\fbox{#2}}}
\newcommand{\fakebox}[1]{\tn{$\ulcorner$#1$\urcorner$}}
\newcommand{\sq}[4]{\xymatrix{#1\ar[r]\ar[d]&#2\ar[d]\\#3\ar[r]&#4}}
\newcommand{\namecat}[1]{\begin{center}$#1:=$\end{center}}

\def\monOb{\blacktriangle}

\def\ullimit{\ar@{}[rd]|(.25)*+{\Large\lrcorner}}
\def\urlimit{\ar@{}[ld]|(.25)*+{\Large\llcorner}}
\def\lllimit{\ar@{}[ru]|(.25)*+{\Large\urcorner}}
\def\lrlimit{\ar@{}[lu]|(.25)*+{\Large\ulcorner}}
\def\ulhlimit{\ar@{}[rd]|(.3)*+{\diamond}}
\def\urhlimit{\ar@{}[ld]|(.3)*+{\diamond}}
\def\llhlimit{\ar@{}[ru]|(.3)*+{\diamond}}
\def\lrhlimit{\ar@{}[lu]|(.3)*+{\diamond}}
\newcommand{\clabel}[1]{\ar@{}[rd]|(.5)*+{#1}}
\newcommand{\TriRight}[7]{\xymatrix{#1\ar[dr]_{#2}\ar[rr]^{#3}&&#4\ar[dl]^{#5}\\&#6\ar@{}[u] |{\Longrightarrow}\ar@{}[u]|>>>>{#7}}}
\newcommand{\TriLeft}[7]{\xymatrix{#1\ar[dr]_{#2}\ar[rr]^{#3}&&#4\ar[dl]^{#5}\\&#6\ar@{}[u] |{\Longleftarrow}\ar@{}[u]|>>>>{#7}}}
\newcommand{\TriIso}[7]{\xymatrix{#1\ar[dr]_{#2}\ar[rr]^{#3}&&#4\ar[dl]^{#5}\\&#6\ar@{}[u] |{\Longleftrightarrow}\ar@{}[u]|>>>>{#7}}}


\newcommand{\arr}[1]{\ar@<.5ex>[#1]\ar@<-.5ex>[#1]}
\newcommand{\arrr}[1]{\ar@<.7ex>[#1]\ar@<0ex>[#1]\ar@<-.7ex>[#1]}
\newcommand{\arrrr}[1]{\ar@<.9ex>[#1]\ar@<.3ex>[#1]\ar@<-.3ex>[#1]\ar@<-.9ex>[#1]}
\newcommand{\arrrrr}[1]{\ar@<1ex>[#1]\ar@<.5ex>[#1]\ar[#1]\ar@<-.5ex>[#1]\ar@<-1ex>[#1]}

\newcommand{\To}[1]{\xrightarrow{#1}}
\newcommand{\Too}[1]{\xrightarrow{\ \ #1\ \ }}
\newcommand{\From}[1]{\xleftarrow{#1}}
\newcommand{\Fromm}[1]{\xleftarrow{\ \ #1\ \ }}

\newcommand{\Adjoint}[4]{\xymatrix@1{#2 \ar@<.5ex>[r]^-{#1} & #3 \ar@<.5ex>[l]^-{#4}}}
\newcommand{\adjoint}[4]{\xymatrix{#1\taking #2\ar@<.5ex>[r]& #3\hspace{1pt}:\hspace{-2pt} #4\ar@<.5ex>[l]}}

\def\id{\tn{id}}
\def\Top{{\bf Top}}
\def\Cat{{\bf Cat}}
\def\Oprd{{\bf Oprd}}
\def\Str{{\bf Str}}
\def\Mon{{\bf Mon}}
\def\Grp{{\bf Grp}}
\def\Grph{{\bf Grph}}
\def\Type{{\bf Type}}
\def\Supp{{\bf Supp}}
\def\Dist{{\bf Dist}}
\def\Vect{{\bf Vect}}
\def\Kls{{\bf Kls}}
\def\Prop{{\bf Prop}}
\def\FLin{{\bf FLin}}
\def\Set{{\bf Set}}
\def\Sets{{\bf Sets}}
\def\PrO{{\bf PrO}}
\def\Star{{\bf Star}}
\def\Cob{{\bf Cob}}
\def\Qry{{\bf Qry}}
\def\set{{\text \textendash}{\bf Set}}
\def\sSet{{\bf sSet}}
\def\sSets{{\bf sSets}}
\def\Grpd{{\bf Grpd}}
\def\Pre{{\bf Pre}}
\def\Shv{{\bf Shv}}
\def\Rings{{\bf Rings}}
\def\bD{{\bf \Delta}}
\def\dispInt{\parbox{.1in}{$\int$}}
\def\bhline{\Xhline{2\arrayrulewidth}}
\def\bbhline{\Xhline{2.5\arrayrulewidth}}
\def\bbbhline{\Xhline{3\arrayrulewidth}}


\def\colim{\mathop{\tn{colim}}}
\def\hocolim{\mathop{\tn{hocolim}}}

\def\mcA{\mc{A}}
\def\mcB{\mc{B}}
\def\mcC{\mc{C}}
\def\mcD{\mc{D}}
\def\mcE{\mc{E}}
\def\mcF{\mc{F}}
\def\mcG{\mc{G}}
\def\mcH{\mc{H}}
\def\mcI{\mc{I}}
\def\mcJ{\mc{J}}
\def\mcK{\mc{K}}
\def\mcL{\mc{L}}
\def\mcM{\mc{M}}
\def\mcN{\mc{N}}
\def\mcO{\mc{O}}
\def\mcP{\mc{P}}
\def\mcQ{\mc{Q}}
\def\mcR{\mc{R}}
\def\mcS{\mc{S}}
\def\mcT{\mc{T}}
\def\mcU{\mc{U}}
\def\mcV{\mc{V}}
\def\mcW{\mc{W}}
\def\mcX{\mc{X}}
\def\mcY{\mc{Y}}
\def\mcZ{\mc{Z}}

\def\undsc{\rule{2mm}{0.4pt}}
\def\Loop{{\mcL oop}}
\def\LoopSchema{{\parbox{.5in}{\fbox{\xymatrix{\LMO{s}\ar@(l,u)[]^f}}}}}

\newtheorem{theorem}[subsubsection]{Theorem}
\newtheorem{lemma}[subsubsection]{Lemma}
\newtheorem{proposition}[subsubsection]{Proposition}
\newtheorem{corollary}[subsubsection]{Corollary}
\newtheorem{fact}[subsubsection]{Fact}

\theoremstyle{remark}
\newtheorem{remark}[subsubsection]{Remark}
\newtheorem{example}[subsubsection]{Example}
\newtheorem{warning}[subsubsection]{Warning}
\newtheorem{question}[subsubsection]{Question}
\newtheorem{guess}[subsubsection]{Guess}
\newtheorem{answer}[subsubsection]{Answer}
\newtheorem{construction}[subsubsection]{Construction}
\newtheorem{rules}[subsubsection]{Rules of good practice}
\newtheorem{exc}[subsubsection]{Exercise}
%\newenvironment{exercise}{\begin{exc}}{\hspace*{\fill}$\lozenge$\end{exc}}
\newtheorem{app}[subsubsection]{Application}
\newenvironment{application}{\begin{app}}{\hspace*{\fill}$\lozenge\lozenge$\end{app}}

%\newenvironment{exercise}{\addtocounter{theorem}{1}\vspace{.1in}\begin{sloppypar}\noindent{\em Exercise}\;\arabic{chapter}.\arabic{section}.\arabic{subsection}.\arabic{theorem}.}{\end{sloppypar}\vspace{.1in}}

\newenvironment{slogan}{\addtocounter{subsubsection}{1}\vspace{.1in}\begin{sloppypar}\noindent{\em Slogan}\;\arabic{chapter}.\arabic{section}.\arabic{subsection}.\arabic{subsubsection}. \begin{quote}``\slshape}{"\end{quote}\end{sloppypar}\vspace{.1in}}

%\newenvironment{application}{\addtocounter{subsubsection}{1}\vspace{.1in}\begin{sloppypar}\noindent{\em Application}\;\arabic{chapter}.\arabic{section}.\arabic{subsection}.\arabic{subsubsection}. \begin{quote}}{\end{quote}\end{sloppypar}\vspace{.1in}}
\makeatletter\let\c@figure\c@equation\makeatother

\theoremstyle{definition}
\newtheorem{definition}[subsubsection]{Definition}
\newtheorem{notation}[subsubsection]{Notation}
\newtheorem{conjecture}[subsubsection]{Conjecture}
\newtheorem{postulate}[subsubsection]{Postulate}


%\newtheorem{theorem}{Theorem}[subsection]
%\newtheorem{lemma}[theorem]{Lemma}
%\newtheorem{proposition}[theorem]{Proposition}
%\newtheorem{corollary}[theorem]{Corollary}
%\newtheorem{fact}[theorem]{Fact}
%
%\theoremstyle{remark}
%\newtheorem{remark}[theorem]{Remark}
%\newtheorem{example}[theorem]{Example}
%\newtheorem{warning}[theorem]{Warning}
%\newtheorem{question}[theorem]{Question}
%\newtheorem{guess}[theorem]{Guess}
%\newtheorem{answer}[theorem]{Answer}
%\newtheorem{construction}[theorem]{Construction}
%\newtheorem{rules}[theorem]{Rules of good practice}
%\newtheorem{exc}[theorem]{Exercise}
%%\newenvironment{exercise}{\addtocounter{theorem}{1}\vspace{.1in}\begin{sloppypar}\noindent{\em Exercise}\;\arabic{chapter}.\arabic{section}.\arabic{subsection}.\arabic{theorem}.}{\end{sloppypar}\vspace{.1in}}
%\newenvironment{exercise}{\begin{exc}}{\hspace*{\fill}$\lozenge$\end{exc}}
%
%\theoremstyle{definition}
%\newtheorem{definition}[theorem]{Definition}
%\newtheorem{notation}[theorem]{Notation}
%\newtheorem{conjecture}[theorem]{Conjecture}
%\newtheorem{postulate}[theorem]{Postulate}

\def\Finm{{\bf Fin_{m}}}
\def\Prb{{\bf Prb}}
\def\Prbs{{\wt{\bf Prb}}}
\def\El{{\bf El}}
\def\Gr{{\bf Gr}}
\def\DT{{\bf DT}}
\def\DB{{\bf DB}}
\def\Tables{{\bf Tables}}
\def\Sch{{\bf Sch}}
\def\Fin{{\bf Fin}}
\def\P{{\bf P}}
\def\SC{{\bf SC}}
\def\ND{{\bf ND}}
\def\Poset{{\bf Poset}}


\newcommand{\MainCatLarge}[1]{ 
	\stackrel{#1}{
		\parbox{4.5in}{\fbox{\parbox{4.4in}{\begin{center}\underline{{\tt Employee} manager worksIn $\simeq$ {\tt Employee} worksIn}\hsp  \underline{{\tt Department} secretary worksIn $\simeq$ {\tt Department}}\end{center}~\\\\\\
			\xymatrix@=8pt{&\LTO{Employee}\ar@<.5ex>[rrrrr]^{\tn{worksIn}}\ar@(l,u)[]+<5pt,10pt>^{\tn{manager}}\ar[dddl]_{\tn{first}}\ar[dddr]^{\tn{last}}&&&&&\LTO{Department}\ar@<.5ex>[lllll]^{\tn{secretary}}\ar[ddd]^{\tn{name}}\\\\\\\LTO{FirstNameString}&&\LTO{LastNameString}&~&~&~&\LTO{DepartmentNameString}
			}
		}}}
	}
}
%\CompileMatrices

\setcounter{secnumdepth}{4}
\setcounter{tocdepth}{1}

\def\sub{\begin{itemize}\item}
\def\sexc{\begin{enumerate}[a.)]\setlength{\itemsep}{.1cm}\setlength{\parskip}{.1cm}\item}
\def\next{\item}
\def\endsub{\end{itemize}}
\def\endsexc{\end{enumerate}}
%\ Add Extra Level of Section
\usepackage{titlesec}

\titleformat{\paragraph}
{\normalfont\normalsize\bfseries}{\theparagraph}{1em}{}
\titlespacing*{\paragraph}
{0pt}{3.25ex plus 1ex minus .2ex}{1.5ex plus .2ex}
%change exercise notation
\usepackage{exercise}
%use fullpage
%\usepackage{fullpage}
\renewcounter{Exercise}[subsubsection]% Reset counter every chapter
%add Quiver package for Diagrams
\usepackage{quiver}

%%%%%
\begin{document}
\begin{large}
  \maketitle
  \begin{center}

  \vspace{-0.3in}
  \begin{tabular}{rl}
  Collaborators: & 
  \end{tabular}
  \end{center}

All of this is my own work.

\noindent
\rule{\linewidth}{0.4pt}
\section{Introduction}
\section{The category of sets}
\subsection{Sets and function}
\subsubsection{Sets}\label{sec:sets}
\begin{Exercise}
The set of all sets of $A=\{1,2,3\}$, also called the power set of $A$, is \begin{align*}\mathcal{P}(A)=\{\emptyset,A,1,2,3,\{1,2\},\{1,3\},\{2,3\},\{1,2,3\}\} \end{align*}
\end{Exercise}
\subsubsection{Functions}\label{sec:funcs}
\begin{Exercise}
\sexc
Because each element of the set of photo-receptive cells ($PR$) connects to exactly one element of the set of retino-ganglial cells($RG$) and no elements of $PR$ connect to two elements of $RG$,  the only valid function would be from $PR\rightarrow RG $. 
\next Each neuron forms at least one synapse with other neurons(otherwise it dies). Thus, the set of function-like connections between brain parts is a subset of the set of all biologically possible connections between brain parts.\\ It therefore seems plausible that the connection patterns that exists between other areas of the brain are function-like.
\endsexc
\end{Exercise}

\begin{Exercise}
As depicted in (2.2), the elements of the codomain of $f$ that receive at least one arrows are $y_1,y_2,y_4$. Therefore $im(f)=\{y_1,y_2,y_4\}$.
\end{Exercise}
\begin{Exercise}
\sexc
$A=\{1,2,3,4,5\}$ and $B=\{x,y\}$. \\
Since each element in $A$ has $|B|$ choices, by the product rule, the total number of functions from $A$ to $B$ is \begin{align*}
|\text{Hom}_\text{Set}(A,B)|=\underbrace{|B|\times |B| \times \cdots \times |B|}_{\text{$|A|$ times}} = 2^5 = 32.
\end{align*}
\next Similarly, since each element of $B$ has $|A|$ choices, the total number of functions from $B$ to $A$ is \begin{align*}
|\text{Hom}_\text{Set}(B,A)|=\underbrace{|A|\times |A| \times \cdots \times |A|}_{\text{$|B|$ times}} = 5^2 = 25.
\end{align*}
\endsexc
\end{Exercise}
\begin{Exercise}
\sexc One set $A$ such that for all sets $X$ there's exactly one element in \begin{align*}\text{Hom}_\text{Set}(X,A) \text{\ is \ } A =\emptyset
\end{align*}. 
This is because all sets contain the empty set and the empty set can only be mapped onto itself. 
\next Similarly, one set $B$ such that for all sets $X$ there's exactly one element in $\text{Hom}_\text{Set}(B,X)$ is $B = \emptyset.$
\endsexc
\end{Exercise}
\begin{Exercise}
Let $X$ be a set with cardinal $n$. 
\sexc 
The first element of $X$ can be mapped to $n$ elements of $X$, the second to $(n-1)$, and so on and the $k$-th to $(n-(k-1))$, thus there are 
\begin{align*}
n\times(n-1)\times \cdots \times 1= n!
\end{align*} isomorphisms from $X$ to itself.
\next 
By convention, $0!=1$ so the formula above holds when $n=0$ and $X=\emptyset$.
\endsexc
\end{Exercise}
\begin{Exercise}
There is no one-to-one correspondence between ``types" of the elements of the sets A and B so there's no obvious choice for which element of B will be associated to each element of the codomain of $f$. For this reason isn't one particular ``canonical funcion" $A \rightarrow \{1,2,3,4,5\}$
\end{Exercise} corresponding to $f$.
\begin{Exercise}
Suppose we have found a set $A$ such that for any set $X$, there's an isomorphism of sets
\begin{align*}X \cong \text{Hom}_\text{Set}(A,X)
\end{align*}
then, the two sets must have the same cardinal:
\begin{align*}|X| = |\text{Hom}_\text{Set}(A,X)|
\end{align*}
Because $\text{Hom}_\text{Set}(A,X)$ is the set of all functions from A to X, its cardinal must be $|X|^{|A|}$, hence:
\begin{align*}
|X| = |X|^{|A|}
\end{align*}
From here we can split into 2 cases, namely $|X|<1$ \text{and} $|X|>1$.\\
If $|X|>1$, applying the logarithm to both sides gives:
\begin{align*}
\text{log}|X| &= |A|{\text{log}|X|}\\
 |A| &= 1
\end{align*}
therefore, only sets $A$ with only one element can be picked.\\
If $|X|<1$, then any set $A$ can be picked.\\
Because the desired property must hold for all sets $X$, the only valid choices for $A$ are among sets containing only one element.
\end{Exercise}
\begin{Exercise}
\sexc
If A = \{a,b,c,d\} and $f:\underline{10}\rightarrow A = \{a,b,c,c,b,a,d,d,a,b\}$, then $f(4)=c$.
\next
$s:\underline{7}\rightarrow\mathbb{N}$ given by $s(i)=i^2$ can be written as a sequence 
\begin{align*}
s =(1,4,9,16,25,36,49).
\end{align*}
\endsexc
\end{Exercise}
\begin{Exercise}
\sexc
$|\{5,6,7\}|=3$;
\next
$|\mathbb{N}|=\infty$;
\next 
$|\{n \in \mathbb{N} \mid n \leq 5\}| = 6.$
\endsexc
\end{Exercise}
\subsection{Commutative diagrams}
\subsection{Ologs}
\subsubsection{Types}
\subsubsection{Aspects}
\begin{Exercise}
Olog that captures the parent-child relationship:\\
% https://q.uiver.app/?q=WzAsMyxbMCwxLCJcXGJveGVke1xcdGV4dHthIHBlcnNvbn19Il0sWzEsMCwiXFxib3hlZHtcXHRleHR7bm8gY2hpbGRyZW59fSJdLFsxLDIsIlxcYm94ZWR7XFx0ZXh0e2EgY2hpbGR9fSJdLFswLDEsIiAgXFx0ZXh0e21heSBoYXZlfSJdLFswLDIsIlxcdGV4dHttYXkgaGF2ZX0iLDJdXQ==
\begin{tikzcd}
	& {\boxed{\text{no children}}} \\
	{\boxed{\text{a person}}} \\
	& {\boxed{\text{a child}}}
	\arrow["{  \text{can have}}", from=2-1, to=1-2]
	\arrow["{\text{can have}}"', from=2-1, to=3-2]
\end{tikzcd}
\end{Exercise}
\begin{Exercise}
Olog for human nuclear biological families:\\
% https://q.uiver.app/?q=WzAsOSxbNSwyLCJcXGJveGVke1xcdGV4dHthIHBlcnNvbn19Il0sWzAsMCwiXFxib3hlZHtcXHRleHR7YSBtYW59fSJdLFswLDQsIlxcYm94ZWR7XFx0ZXh0e2Egd29tYW59fSJdLFszLDIsIlxcYm94ZWR7XFx0ZXh0e2EgY2hpbGR9fSJdLFsxLDIsIlxcYm94ZWR7XFx0ZXh0e2EgcGFyZW50fX0iXSxbMyw0XSxbMSwxLCJcXGNoZWNrbWFyayJdLFsxLDMsIlxcY2hlY2ttYXJrIl0sWzIsMl0sWzEsMCwiXFx0ZXh0e2lzfSJdLFsyLDAsIlxcdGV4dHtpc30iXSxbMSw0LCJcXHRleHR7aXN9Il0sWzIsNCwiXFx0ZXh0e2lzfSJdLFs0LDMsIlxcdGV4dHtoYXMgYX0iXSxbMywwLCJcXHRleHR7aXN9Il1d
\begin{tikzcd}
	{\boxed{\text{a man}}} \\
	& \checkmark \\
	& {\boxed{\text{a parent}}} & {} & {\boxed{\text{a child}}} && {\boxed{\text{a person}}} \\
	& \checkmark \\
	{\boxed{\text{a woman}}} &&& {}
	\arrow["{\text{is}}", from=1-1, to=3-6]
	\arrow["{\text{is}}", from=5-1, to=3-6]
	\arrow["{\text{is}}", from=1-1, to=3-2]
	\arrow["{\text{is}}", from=5-1, to=3-2]
	\arrow["{\text{has a}}", from=3-2, to=3-4]
	\arrow["{\text{is}}", from=3-4, to=3-6]
\end{tikzcd}
\end{Exercise}
\begin{Exercise}
Given x, an operational land-line phone, consider the following. We know that $x$ is an operational land-line phone,\\
which is assigned to a phone number, which has an area code, which corresponds to a region\\
that we'll call $P(x)$.\\
We also know that $x$ is an operational land-line phone,
which is a physical phone, which is currently located in a region\\
that we'll call $Q(x)$.\\
Fact: whenever $x$ is an operational land-line phone, we will have $P(x)=Q(x)$.\\ 
\end{Exercise}
\begin{Exercise}
If the box  ``an operational land-line phone" was replaced with the box ``an operational mobile phone", the region in which it is currently located would not not always match the region corresponding to the mobile phone's phone number's area code and diagram would not commute.
\end{Exercise}
\begin{Exercise}

\end{Exercise}
\end{large}
\end{document}
